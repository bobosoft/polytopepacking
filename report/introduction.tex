\section{Introduction}
Polytope Packing is the problem of placing a given set of
polytopes into a parallelepiped of given length and width
with the goal of finding the minimal possible height that
avoids polytopes collision (polytopes may touch but they
cannot compenetrate). 

The problem has been studied for instance by 
Stoyan et. al in~\cite{sto03}, which we use as
a reference and comparison for this report. More related
work can be found in the aforementioned paper and others
by the same authors.

Our approach is a plain encoding into an SMT2~\cite{SMTLIB} 
formula. SMT, Satisfiability Modulo Theories, is an area of 
research that combines efficient SAT-Solving and domain-specific
decision procedures to build efficient tools that could
reason about, for instance, arbitraty boolean combinations 
of linear arithmetic costraints. Efficient SMT-Solvers are
available off-the-shelf and under continuos improvement. 
Our encoding into SMT exploits the notion of Minkowski sum 
to formally describe concepts such as ``polytope intersection''.

Therefore, the approach can be summarized as follows: 
take a set of polytope descriptions, encode the problem
into SMT2, execute an SMT-Solver to find a solution (if any
exists), read the solution and translate it back to
coordinates that describes the polytopes placement.
